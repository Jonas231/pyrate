\documentclass[a4paper,12pt]{article}
\usepackage[utf8]{inputenc}
\usepackage{amsmath}
\usepackage{amssymb}
\usepackage{dsfont}

%opening
\title{Anisotropic Refraction Formula}

\newcommand{\vct}[1]{{\bf #1}}

\begin{document}

\maketitle

\section{General dispersion formula}

Let's start with the propagator formula for the electrical field after a plane wave decomposition of the Maxwell equations in
source free notation.
\begin{align}
 (-\vct{k}^2 \delta_{ij} + k_i k_j + \omega^2 \mu_0 \varepsilon_{ij}) E_j &= 0\,.
\end{align}
Now pull out $\varepsilon_0$ from $\varepsilon_{ij} = \varepsilon_0 \epsilon_{ij}$ and $\tfrac{\omega}{c}$ 
from $k_i = \tfrac{\omega}{c} \tilde{k}_i$. After using $\mu_0 \varepsilon_0 = \tfrac{1}{c^2}$, this leads to
\begin{align}
 \frac{\omega^2}{c^2} (-\vct{\tilde{k}}^2 \delta_{ij} + \tilde{k}_i \tilde{k}_j + \epsilon_{ij}) E_j &= 0\,.
\end{align}
Therefore the dispersion relation for $\tilde{k}_i$ -- if $\omega \ne 0$ -- is given by (in the following neglecting the tilda
over $k_i$)
\begin{align}
 P(\vct{k}) = \det(-\vct{k}^2 \delta_{ij} + k_i k_j + \epsilon_{ij}) &= 0\,.
\end{align}
The determinant can be calculated in closed form and is given by (RTFM :-))
\begin{align}
\label{eq:det}
 P(\vct{k}) &= -\det\epsilon + (\epsilon_{ij} k_i k_j)(\text{tr }\epsilon - \vct{k}^2) - (k_i \epsilon_{i\ell} \epsilon_{\ell j} k_j)\,.
\end{align}

\section{Decomposition}
Now decomposing $k_i$ relative to a plane in the formula \eqref{eq:det} into an in-plane component $k_i^\parallel$ and an 
out-of-plane component $\xi\,n_i$, namely $k_i = k_i^\parallel + \xi\,n_i$, where $\vct{n}^2 = 1$ and 
$\vct{n}\cdot\vct{k}^\parallel = 0$.
This leads to:
\begin{align}
 P(\vct{k}^\parallel, \xi) &= -\det\epsilon\nonumber\\& 
  + [(\epsilon_{ij} k^\parallel_i k^\parallel_j) 
    + \xi (\epsilon_{ij} + \epsilon_{ji}) k_i^\parallel n_j 
    + \xi^2 (\epsilon_{ij} n_i n_j)](\text{tr }\epsilon - \vct{k}^2_\parallel - \xi^2)\nonumber\\&
  - [(k^\parallel_i \epsilon_{i\ell} \epsilon_{\ell j} k^\parallel_j)
     + \xi (\epsilon_{i\ell} \epsilon_{\ell j} + \epsilon_{j\ell} \epsilon_{\ell i}) k_i^\parallel n_j
     + \xi^2 (n_i \epsilon_{i\ell} \epsilon_{\ell j} n_j)]\,.
\end{align}
or in more simplified form:
\begin{align}
 P &= a_0 + (a_1 + \xi a_2 + \xi^2 a_3) (a_4 - \xi^2) + (a_5 + \xi a_6 + \xi^2 a_7)\,,\nonumber\\
 &= a_0 + a_1 a_4 + a_5 + (a_4 a_2 + a_6) \xi + (a_3 a_4 + a_7 - a_1) \xi^2 - a_2 \xi^3 - a_3 \xi^4\,.
\end{align}
where
\begin{subequations}
\begin{align}
  a_0 &= -\det\epsilon\,,\\
  a_1 &= (\epsilon_{ij} k^\parallel_i k^\parallel_j)\,,\\
  a_2 &= (\epsilon_{ij} + \epsilon_{ji}) k_i^\parallel n_j\,,\\
  a_3 &= (\epsilon_{ij} n_i n_j)\,,\\
  a_4 &= \text{tr }\epsilon\,,\\
  a_5 &= (k^\parallel_i \epsilon_{i\ell} \epsilon_{\ell j} k^\parallel_j)\,,\\
  a_6 &= (\epsilon_{i\ell} \epsilon_{\ell j} + \epsilon_{j\ell} \epsilon_{\ell i}) k_i^\parallel n_j\,,\\
  a_7 &= (n_i \epsilon_{i\ell} \epsilon_{\ell j} n_j)\,.
\end{align}
\end{subequations}
As one can see, some of the $a_I$ could be rearranged into powers of $\xi$, but all of them are irreducible scalar quantities
built from $\epsilon_{ij}$, $k_i^\parallel$, and $n_i$. Maybe we should have a look at the discriminant formulas of such a
quartic equation. 

For this reason we introduce a new form first by dividing by $-a_3$ (which is even for very exotic materials $\ne0$)
\begin{align}
 \bar{P} = -\frac{P}{a_3} &= -\frac{a_0 + a_1 a_4 + a_5}{a_3} - \frac{a_4 a_2 + a_6}{a_3} \xi - \frac{a_3 a_4 + a_7 - a_1}{a_3} \xi^2 + \frac{a_2}{a_3} \xi^3 + \xi^4\,.
\end{align}

Usually, one would now introduce a linear Tschirnhaus transform ($\xi = u - \tfrac{a_2}{4 a_3}$) to achieve the depressed form of the quartic for $u$. 
But since the coefficients get more complicated, maybe this is not the way to go. For us it is more interesting to investigate the zeros structure of 
the polynomial by utilizing long division. Basis for this investigation is that
\begin{itemize}
 \item A polynomial $p(\xi)$ has no double zeros iff $\text{GCD}(p(\xi), p'(\xi)) = 1$
 \item A polynomial $p(\xi)$ has no triple zeros iff $\text{GCD}(p(\xi), p''(\xi)) = 1$
 \item A polynomial $p(\xi)$ has no quadruple zeros iff $\text{GCD}(p(\xi), p'''(\xi)) = 1$
\end{itemize}
Those arguments can easily be seen for
\begin{align}
 p(\xi) = (\xi - \xi_1)(\xi - \xi_2)(\xi - \xi_3)(\xi - \xi_4)\,,
\end{align}
where the $\xi_I$ are the zeros of the polynomial. If now $p(\xi)$ has a double zero, e.g., $\xi_1$, then the factor $(\xi - \xi_1)$
would appear also in the first order derivative. Therefore the GCD would be $(\xi - \xi_1) \ne 1$. Same goes for triple zeros and
quadruple zeros where the higher derivatives always have a factor in common with the polynomial. We demonstrate this for the third
derivative, showing under which conditions there are quadruple zeros.

For our polynomial, where the coefficients are now abbreviated by
\begin{align}
 \bar{P} &= \xi^4 + A \xi^3 + B \xi^2 + C \xi + D\,,
\end{align}
the third derivative for example is given by
\begin{align}
 \bar{P}''' &= 24 \xi + 6 A = 6 (4 \xi + A)\,.
\end{align}
{\tiny
Dividing:
\begin{align}
 (\ast) &= \frac{1}{4} \xi^3 + \frac{3}{16} A \xi^2 + \left(\frac{B}{4} - \frac{3}{64} A^2\right) \xi + \frac{1}{4}\left[C - A \left(\frac{B}{4} - \frac{3}{64} A^2\right)\right]
\end{align}
\begin{align}
\begin{array}{cccccccccc}
 (\xi^4 & + A \xi^3 & + B \xi^2 & + C \xi & + D) & : & (4 \xi & + A) & = & (\ast)\\
 (\xi^4 & + \frac{A}{4} \xi^3) & &         &      &   &        &      &   & \\
 \hline
	& (\frac{3}{4} A \xi^3 & + B \xi^2 & + C \xi & + D) & & &  & & \\
	& (\frac{3}{4} A \xi^3 & + \frac{3}{16} A^2 \xi^2) & & & & &  & & \\
	\hline
	& & (\left(B - \frac{3}{16} A^2\right) \xi^2 & + C \xi & + D) & & &  & &  \\
	& & (\left(B - \frac{3}{16} A^2\right) \xi^2 & + A \left(\frac{B}{4} - \frac{3}{64} A^2\right) \xi) & & & &  & &  \\
	\hline
	& & & (\left[C - A \left(\frac{B}{4} - \frac{3}{64} A^2\right)\right] \xi & +D) & & &  & &  \\
	& & & (\left[C - A \left(\frac{B}{4} - \frac{3}{64} A^2\right)\right] \xi & \frac{A}{4}\left[C - A \left(\frac{B}{4} - \frac{3}{64} A^2\right)\right]) & & &  & &  \\
	\hline
	& & & & D - \frac{A}{4}\left[C - A \left(\frac{B}{4} - \frac{3}{64} A^2\right)\right] & & &  & &
\end{array}
\end{align}
}
This means the division remainder is vanishing (implying there are quadruple zeros) if
\begin{align}
 D - \frac{A}{4}\left[C - \frac{A}{4} \left(B - \frac{3}{16} A^2\right)\right] = 0\,,
\end{align}
where
\begin{subequations}
\begin{align}
 A &= \frac{a_2}{a_3} = \dots\,,\\
 B &= - a_4 - \frac{a_7 - a_1}{a_3} = \dots\,,\\
 C &= a_4 A -\frac{a_6}{a_3} = \dots\,,\\
 D &= -\frac{a_0 + a_1 a_4 + a_5}{a_3} = \dots\,.
\end{align}
\end{subequations}






\end{document}
