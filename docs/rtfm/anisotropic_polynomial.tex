\documentclass[a4paper,12pt]{article}
\usepackage[utf8]{inputenc}
\usepackage{amsmath}
\usepackage{amssymb}
\usepackage{dsfont}

%opening
\title{Anisotropic Refraction Formula}

\newcommand{\vct}[1]{{\bf #1}}

\begin{document}

\maketitle

\section{General dispersion formula}

Let's start with the propagator formula for the electrical field after a plane wave decomposition of the Maxwell equations in
source free notation.
\begin{align}
 (-\vct{k}^2 \delta_{ij} + k_i k_j + \omega^2 \mu_0 \varepsilon_{ij}) E_j &= 0\,.
\end{align}
Now pull out $\varepsilon_0$ from $\varepsilon_{ij} = \varepsilon_0 \epsilon_{ij}$ and $\tfrac{\omega}{c}$ 
from $k_i = \tfrac{\omega}{c} \tilde{k}_i$. After using $\mu_0 \varepsilon_0 = \tfrac{1}{c^2}$, this leads to
\begin{align}
 \frac{\omega^2}{c^2} (-\vct{\tilde{k}}^2 \delta_{ij} + \tilde{k}_i \tilde{k}_j + \epsilon_{ij}) E_j &= 0\,.
\end{align}
Therefore the dispersion relation for $\tilde{k}_i$ -- if $\omega \ne 0$ -- is given by (in the following neglecting the tilda
over $k_i$)
\begin{align}
 P(\vct{k}) = \det(-\vct{k}^2 \delta_{ij} + k_i k_j + \epsilon_{ij}) &= 0\,.
\end{align}
The determinant can be calculated in closed form and is given by (RTFM :-))
\begin{align}
\label{eq:det}
 P(\vct{k}) &= -\det\epsilon + (\epsilon_{ij} k_i k_j)(\text{tr }\epsilon - \vct{k}^2) - (k_i \epsilon_{i\ell} \epsilon_{\ell j} k_j)\,.
\end{align}

\section{Decomposition}
Now decomposing $k_i$ relative to a plane in the formula \eqref{eq:det} into an in-plane component $k_i^\parallel$ and an 
out-of-plane component $\xi\,n_i$, namely $k_i = k_i^\parallel + \xi\,n_i$, where $\vct{n}^2 = 1$ and 
$\vct{n}\cdot\vct{k}^\parallel = 0$.
This leads to:
\begin{align}
 P(\vct{k}^\parallel, \xi) &= -\det\epsilon\nonumber\\& 
  + [(\epsilon_{ij} k^\parallel_i k^\parallel_j) 
    + \xi (\epsilon_{ij} + \epsilon_{ji}) k_i^\parallel n_j 
    + \xi^2 (\epsilon_{ij} n_i n_j)](\text{tr }\epsilon - \vct{k}^2_\parallel - \xi^2)\nonumber\\&
  - [(k^\parallel_i \epsilon_{i\ell} \epsilon_{\ell j} k^\parallel_j)
     + \xi (\epsilon_{i\ell} \epsilon_{\ell j} + \epsilon_{j\ell} \epsilon_{\ell i}) k_i^\parallel n_j
     + \xi^2 (n_i \epsilon_{i\ell} \epsilon_{\ell j} n_j)]\,.
\end{align}
or in more simplified form:
\begin{align}
 P &= a_0 + (a_1 + \xi a_2 + \xi^2 a_3) (a_4 - \xi^2) + (a_5 + \xi a_6 + \xi^2 a_7)\,.
\end{align}
As one can see, some of the $a_I$ could be rearranged into powers of $\xi$, but all of them are irreducible scalar quantities
built from $\epsilon_{ij}$, $k_i^\parallel$, and $n_i$. Maybe we should have a look at the discriminant formulas of such a
quartic equation.

\end{document}
