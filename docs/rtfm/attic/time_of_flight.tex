\documentclass[12pt,a4paper,twoside,openright,BCOR10mm,headsepline,titlepage,abstracton,chapterprefix,final]{scrreprt}

\usepackage{ae}
\usepackage[ngerman, english]{babel}
%\usepackage{SIunits}

\usepackage{amsmath}
\usepackage{amssymb}
\usepackage{amsfonts}
\usepackage{xcolor}
\usepackage{setspace}
\usepackage{dsfont}

% load hyperref as the last package to avoid redefinitions of e.g. footnotes after hyperref invokation

\RequirePackage{ifpdf}  % flag for pdf or dvi backend
\ifpdf
  \usepackage[pdftex]{graphicx}
  \usepackage[pdftitle={RTFM on Imaging Theory and Basics of Optical Raytracing},%
              pdfsubject={},%
              pdfauthor={M. Esslinger, J. Hartung, U. Lippmann},%
              pdfkeywords={},%
              bookmarks=true,%
%              colorlinks=true,%
              urlcolor=blue,%
              pdfpagelayout=TwoColumnRight,%
              pdfpagemode=UseNone,%
              pdfstartview=Fit,%
	      pdfpagelabels,
              pdftex]{hyperref}
\else
  \usepackage[dvips]{graphicx}
  \usepackage[colorlinks=false,dvips]{hyperref}
\fi
%\DeclareGraphicsRule{.jpg}{eps}{.jpg}{`convert #1 eps:-}

\usepackage{ae}
%\usepackage[ngerman, english]{babel}

%\usepackage{SIunits}
\newcommand\elementarycharge{\textrm{e}}
\newcommand\sccm{\textrm{sccm}}
\newcommand\mbar{\milli\textrm{bar}}


\usepackage{amsmath}
%\usepackage{amssymb}
\usepackage{setspace}

%\widowpenalty = 1000


\newcommand*{\doi}[1]{\href{https://doi.org/\detokenize{#1}}{doi: \detokenize{#1}}}

\newcommand\Vector[1]{{\mathbf{#1}}}
%\newcommand\Vector[1]{{\vec{#1}}}

\newcommand\vacuum{0}

\newcommand\location{r}
\newcommand\Location{\Vector{r}}


\newcommand\wavenumber{k}
\newcommand\vacuumWavenumber{\wavenumber_{\vacuum}}
\newcommand\Wavevector{\Vector{\wavenumber}}

\newcommand\Nabla{\Vector{\nabla}}
\newcommand\Laplace{\Delta}
\newcommand\timederivative[1]{\dot{{#1}}}
\newcommand\Tensor[1]{\hat{#1}}
\newcommand\conjugate[1]{\overline{#1}}
\newcommand\transpose[1]{#1^{T}}
\newcommand\Norm[1]{\left| #1 \right|}
\newcommand{\ket}[1]{\left\vert{#1}\right\rangle}
\newcommand{\bra}[1]{\left\langle{#1}\right\vert}
\newcommand{\braket}[2]{\left\langle{#1}\vert{#2}\right\rangle}
\newcommand{\bracket}[1]{\left\langle{#1}\right\rangle}

\newcommand{\scpm}[2]{(#1\,\cdot\,#2)}

\newcommand\unittensor{\mathds{1}}

\newcommand\Greenfunction{\Tensor{G}}

\newcommand\scalarEfield{E}
\newcommand\scalarBfield{B}
\newcommand\scalarHfield{H}
\newcommand\scalarDfield{D}
\newcommand\Efield{\Vector{\scalarEfield}}
\newcommand\Bfield{\Vector{\scalarBfield}}
\newcommand\Hfield{\Vector{\scalarHfield}}
\newcommand\Dfield{\Vector{\scalarDfield}}

\newcommand\permeability{\Tensor{\scalarpermeability}}
\newcommand\vacuumpermeability{\scalarpermeability_{\vacuum}}
\newcommand\scalarpermeability{\mu}
\newcommand\scalarrelativepermeability{\mu_{rel}}
\newcommand\relativepermeability{\Tensor{\mu}_{rel}}

\newcommand\permittivity{\Tensor{\scalarpermittivity}}
\newcommand\vacuumpermittivity{\scalarpermittivity_{\vacuum}}
\newcommand\scalarrelativepermittivity{\epsilon}
\newcommand\relativepermittivity{\Tensor{\scalarrelativepermittivity}}
\newcommand\scalarpermittivity{\varepsilon}

\newcommand\conductivity{\Tensor{\sigma}}
\newcommand\susceptibility{\Tensor{\chi}}
\newcommand\currentdensity{\Vector{j}}
\newcommand\chargedensity{\rho}
\newcommand\PoyntingVector{\Vector{S}}

\newcommand\ordi{\text{ord}}
\newcommand\eo{\text{eo}}

\newcommand\materialone{I}
\newcommand\materialtwo{{II}}

\newcommand{\kpa}[1]{{\wavenumber_{\parallel#1}}}
\newcommand\tr{\text{tr}}

\newcommand{\timeavg}[1]{{\langle\,#1\,\rangle}}

\newcommand{\remark}[1]{{\color{red}$\blacksquare$}\footnote{{\color{red}#1}}}
\newcommand{\chk}[1]{\color{green}{$\checkmark$#1}}

\newcommand{\orderof}[1]{\mathcal{O}(#1)}

\newcommand\ppol{p}
\newcommand\spol{s}
\newcommand\normconst{N}

\newcommand\kilogram{\textrm{kg}}
\newcommand\meter{\textrm{m}}
\newcommand\second{\textrm{s}}
\newcommand\ampere{\textrm{A}}
\newcommand\volt{\textrm{V}}
\newcommand\watt{\textrm{W}}
\newcommand\tesla{\textrm{T}}

\newcommand\totald{\textrm{d}}

\begin{document}

\section{Time of Flight}
\subsection{Introduction}
We are interested in the speed a signal (information) can travel through a medium.
Equivalently, we can determine the time a signal needs to propagate through a medium.
To do so, we model an experimental setup and then simulate it.

We consider a strictly monochromatic laser. 
We use a modulator, for example amplitude or phase modulator, 
to imprint a signal on laser carrier frequency.
The modulated light then propagates through the medium we want to test.
After passing the medium, the light is detected.
The time of flight is the time the signal takes to pass the medium.

\subsection{Details}
\begin{itemize}
 \item We restrict ourselves to digital data transmission, one bit at a time.
       0 and 1 are represented by two distinguishable modulation patterns.
       We do not want to consider modulation techniques where several bits are transmitted at once,
       with a multitude of modulation patterns.
       In radio technology, these patterns are called \emph{symbols}.
 \item We assume the modulator is unable to instantly switch between the two states.
 \item We model the modulator with digital input by a function generator $f$, 
       and a signal input $g$.
 \item For an input signal of $g=0$, the function generator shall produce the modulation pattern 0.
       For an input signal of $g=1$, the function generator shall produce the modulation pattern 1.
       The function generator output signal $f(g)$ shall be continuous 
       for all continuous inputs $g(t)$.
 \item We model finite switching times by a smooth transition from $g=0$ to 1.
 \item We are interested in a single mode at a time. 
       The polarisation of the laser has to be chosen in a way 
       that the light is in an eigenstate of the medium's dispersion equation.
       This way, the incident light does not split up in several beams, as it could in anisotropic media. 
 \item Also the modulator must be built 
       such that at any time all the light stays on the same solution branch of the dispersion.
       Through the modulator, the light entering the medium is no more strictly monochromatic. 
       Also frequencies off the monochromatic laser frequency need to be considered.
 \item The time of successful signal detection is when the detector can distinguish 
       a scenario with a constant value of $g$ from a scenario with $g$ changing from 0 to 1 or vice versa. 
 \item We can build a detector to detect quantities like, for example, intensity or interferometric phase.
\end{itemize}

The electric field after passing the modulator is:
\begin{eqnarray}
 \Efield(\Location, t) = \Efield_0 \exp (i (\Wavevector \Location - \omega_0 t)) f(t)
\end{eqnarray}
Where we can choose the modulation, for example, like:
\begin{eqnarray}
 \textrm{On/off amplitude modulation scheme}&& f = g\\
 \textrm{Lock-in detection amplitude modulation scheme}&& f = 1 - \frac{g}{2} (1 + \sin{\Omega t}) \\
 \textrm{$\lambda/4$ phase shift for interferometric detection}&& f = \exp\left( \frac{i \pi}{2} g \right) \\
 \textrm{with}&& 0 \leq g \leq 1
\end{eqnarray}


\subsection{The Impossibility of Pre-Detection Proof Signals with Finite Frequency Support}
We want an input signal with the following properties:
\begin{itemize}
 \item The signal $g$ shall be defined at all times.
       We do not limit the signal to the interval $[0,1]$, but allow temporary overshooting.\\
       $g:\mathbb{R}\rightarrow \mathbb{R}$
 \item The data transmission (change of $g$) shall start earliest at $t=0$.
       Before $t=0$, the signal shall be exactly zero.\\
       $g(t)=0 \,\forall t\leq0$
 \item There shall be a finite time interval where the signal is nonzero.\\
       $\exists t_1\geq0, \tau>0: g(t) \geq 1 \forall t \in [t_1, t_1+\tau]$
 \item $g$ shall be continuous.
 \item The Fourier transform of $g$ shall exist.
 \item The Fourier transform $\tilde{g}$ shall be zero 
       for all frequencies greater than a finite maximum frequency $\omega_{max}$.
       The electric field after the modulator is 
       the product of the the monochromatic laser oscillation and the switching function.
       There will be sum- and difference frequencies in the modulated signal.
       We would like to stay in the near-monochromatic case, 
       where only frequencies close to the laser frequency contribute to the electric field.\\
       $\exists \omega_{max}>0 : \tilde{g}(\omega) = 0 \,\forall |\omega| > \omega_{max}$
\end{itemize}

\paragraph{Hypothesis:} There is no function $g$ which fulfills all of the above requirements.

\paragraph{Proof by contradiction:}
Let $g$ be a function that fulfills all of the above requirements.
First we show that $g$ cannot be infinitely continuously differentiable.
If $g$ was infinitely differentiable, then it could be represented by a Taylor series.
In the point $t=0$, all derivatives of $g$ must be zero, as the function is zero at all times $t<0$.
Therefore, all Taylor series terms for a series around $t=0$ are zero 
and thus $g$ would be zero at all times.
However, we want $g$ to be nonzero at at least one instance in time.
So we conclude that either $g$ itself or a derivative of $g$ is discontinuous in at least one point.
We require $g$ to be continuous, so a derivative of $g$ must be discontinuous.

We assume the $n$-th derivative is the lowest order derivative which is discontinuous.
We call this derivative $\partial^n g = g_n$.

TODO: show that $g_n$ cannot have removable discontinuities with finite values.

TODO: show that $g_n$ does not have a finite frequency support.

\begin{eqnarray}
 \nexists \omega_{max}>0 :  \tilde{g}_n(\omega) = 0 \forall |\omega| > \omega_{max}
\end{eqnarray}
We consider the derivative of an arbitrary Fourier transformable function $h$:
\begin{eqnarray}
 h(t) &=& \frac{1}{2\pi} \int \totald\omega\, \tilde{h}(\omega) \exp(i \omega t) \\
 \partial^1 h(t) &=& \frac{1}{2\pi} \int \totald\omega\, i\omega \tilde{h}(\omega) \exp(i \omega t)
\end{eqnarray}
If, at a fixed, nonzero frequency, a Fourier component of $h$ is nonzero, also the Fourier component of its derivative is nonzero $\partial^1 H$.
This also works the other way round: If a derivative has a nonzero Fourier component, its primitive has also a nonzero component at this frequency.
We associate $\partial^n g = \partial^1 h$ and find that also $\partial^{n-1} g$ has no finite frequency support.
From this follows that also $g$ has no finite frequency support.
\begin{eqnarray}
 \nexists \omega_{max}>0 :  \tilde{g}(\omega) = 0 \forall |\omega| > \omega_{max}
\end{eqnarray}
which violates one of the above requirements.

\paragraph{Conclusion:}
\begin{itemize}
 \item Either the signal starts to change at $t\rightarrow-\infty$.
       In this case, it is possible to have a function with finite frequency support.
       The point in time when the transmission starts, however, is hard to define.
       A noiseless detector could sense the faint change of $g$ at an arbitrary time.
 \item Or the signal is zero before the transmission starts, $g(t<0)=0$.
       In this case, the function cannot have a finite frequency support.
       The dispersion must be known at all frequencies up to infinity.
       A noiseless detector could be equipped with a filter that only transmits light at very high frequencies,
       where typical materials become transparent and dispersionless like vacuum.
       Such a detector could always detect signals at the vacuum speed of light.
\end{itemize}


\subsection{Propagation Through the Medium}
We consider the light right after the incoupling into the medium.
We assume the medium permittivity $\permittivity$ is known,
as well as the wavevector $\Wavevector$ and polarisation $\Efield_0$ of the light at the monochromatic carrier frequency $\omega_0$.
We calculate the Poynting vector $\Vector{S}$ of the carrier frequency wave 
and consider a plane perpendicular to the Poynting vector as start plane.
We choose our coordinate system such that the Poynting vector is along the $z$-direction, $\Vector{S} \propto \Vector{e}_z$,
and the ray starts at the origin.
\begin{eqnarray}
 \Efield(x,y,0, t) &=& \Efield_0 \exp (i ( \wavenumber_x x + \wavenumber_y y - \omega_0 t)) f(t)
\end{eqnarray}
where $f(t)$ represents the time-dependent transmission and phase-shift of the modulator.
In isotropic media, the phase velocity points in the same direction as the Poynting vector, 
so in this case $\wavenumber_x = \wavenumber_y = 0$ holds.
We Fourier transform the electric field in the time domain and yield
\begin{eqnarray}
 \Efield(x,y,0, \omega) &=& \Efield_0  \exp (i ( \wavenumber_x x+ \wavenumber_y y)) F(\omega-\omega_0) \label{eq:tof_startplane}\\
 F(\omega) &=& \frac{1}{\sqrt{2\pi}} \int dt f(t) \exp(-i \omega t)
\end{eqnarray}
The detector shall be located on a line from the ray start point $(0,0,0)$ in the direction of the Poynting vector at $(0,0,z_{det})$.
From angular spectrum representation, we derive that for each frequency, the electric field is a single plane wave.
The in-plane components are $\wavenumber_x$,$\wavenumber_y$ of this plane wave are fixed by \eqref{eq:tof_startplane}.
The remaining wavevector component is determined by the dispersion relation.
\begin{eqnarray}
 \Wavevector(\omega) &=& \Wavevector_x + \Wavevector_y + \Wavevector_z(\omega)
\end{eqnarray}
We propagate to the position of the detector
\begin{eqnarray}
\Efield(0, 0, z_{det}, \omega) &=& \Efield_0  \exp (i \wavenumber_z(\omega) z_{det}) F(\omega-\omega_0) 
\end{eqnarray}
We Fourier transform back into time domain and determine the point in time when the signal arrives.
\begin{eqnarray}
\Efield(z_{det}, t) &=& 
  \Efield_0 
  \frac{1}{\sqrt{2\pi}} \int \totald\omega\, 
  F(\omega-\omega_0)
  \exp (i \wavenumber_z(\omega) z_{det})
  \exp ( i \omega t)
\end{eqnarray}
We substitute $\omega = \omega_0 + \Delta \omega$ and yield
\begin{eqnarray}
\Efield(z_{det}, t) &=& 
  \Efield_0 \exp ( i \omega_0 t)
  \cdot
  \underbrace{
      \frac{1}{\sqrt{2\pi}} \int \totald\Delta\omega\, 
      F(\Delta \omega)
      \exp (i \wavenumber_z z_{det})
      \exp ( i \Delta \omega t)
      }_{\textrm{modulation, retarded by medium}}
\nonumber\\
\end{eqnarray}

\section{Old stuff}
\subsection{Attempt 1: Gau\ss\, pulse}
We assume an amplitude modulator which impresses a series of Gau\ss\, pulses on the laser intensity.
\begin{eqnarray}
  f(t) &=& \sum_j \exp\left(- \frac{i(t-t_j)^2}{\tau^2}\right) 
\end{eqnarray}
where $t_j$ are the peak positions and $\tau$ is the pulse duration.
We assume the pulse duration is sufficiently slow compared to the laser carrier frequency 
so that we can treat the problem in a near-monochromatic way.
The peak positions shall be well separated.
Now let's assume we have a noise-free detector and we know the signal consists only of Gau\ss\, pulses.
We measure the signal for some time long before the first peak.

We deconvolute the signal with the known Gau\ss\, shape and anticipate the temporal location of all peaks.
We can do this for measurements at any time before, during, or after the series of pulses.
The problem is that Gau\ss pulses have an infinite extent in time.
Strictly speaking, the signal transmission starts at minus infinity.

From this attempt, we conclude that we may not combine infinitely extended signals and noise-free transmission channels.

\subsection{Attempt 2: Noisy detector}
We assume the same series of Gau\ss\, pulses as above, but this time with some noise on the detected signal.
We determine the detector noise level and define a minimum intensity above the noise level, on which the detector triggers.

The choice of start pulse strength and detector trigger level are arbitrary and influence the measurement result.
There is no fundamental limit for signal-to-noise ratio, as long as there is enough light to suppress shot noise.

\subsection{Attempt 3:}

\begin{itemize}
 \item If we define the peak position as signal position, we also may run into trouble.
       We consider a Sellmeier-oscillator and put our laser close to the resonance.
       The Gau\ss\, pulse will disperse and will be damped.
       If the damping of the slow moving components is stronger than the damping of the fast moving components,
       the center of mass is shifted in positive z direction.
       In some materials, the pulse center of mass can even move faster than the vacuum speed of light.
\end{itemize}

\subsection{Example 2: Heaviside Step}
\begin{itemize}
 \item The Heaviside function has the advantage that the time of switching is unambiguous -- at least at the position of the modulator.
 \item The second advantage is that the function is exactly zero for $t<0$.
       For an observer at the detector position, it is impossible to forsee the time of switching.
 \item The Heaviside function has an infinite spectrum.
       It creates (weak) components at frequencies far from the carrier frequency.
       An observer with a noise-free detector could therefore listen to a frequency far from the carrier frequency,
       where the Sellmeier oscillators have all faded out.
       At this frequency, the material has no dispersion, $n=1$, 
       and the observer would always measure the vacuum speed of light.
\end{itemize}

\subsection{Example 3}
We need a bandwidth-limited step modulation.
\begin{eqnarray}
 f( t < 0 ) &=& \left|0\right> \\
 f( t > \tau) &\approx& f(\infty) \neq \left|0\right> \\
 F(|\omega - \omega_0| > \Omega_{max}) &=& 0
\end{eqnarray}
with finite $\tau$ and $\Omega_{max}$.
Both should be as small as possible, which is a contradiction.

\subsection{lossless media}
\begin{eqnarray}
 \wavenumber_z(\omega) &\in& \mathbb{R} \\
 t_{flight} &=& \left. \frac{\partial \wavenumber_z}{\partial \omega}\right|_{\omega_0} z_{det}
\end{eqnarray}


\end{document}
