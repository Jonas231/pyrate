\documentclass[a4paper,10pt]{revtex4-1}
\usepackage[utf8]{inputenc}
\usepackage{amsmath}

%opening
\title{Eikonal approximation for eigenvalue equation}

\begin{document}

\section{Standard evaluation}

For isotropic media it follows from
\begin{align}
 \partial_i D_i &= 0\,,
\end{align}
with $D_i = \varepsilon \delta_{ij} E_j = \varepsilon E_i$ that
\begin{align}
 \partial_i E_i &= 0\,.
\end{align}

Further from the wave equation
\begin{align}
 (\delta_{ij} \Delta - \partial_i \partial_j + \varepsilon \delta_{ij}) E_j &= 0
\end{align}
it follows the middle term can be neglected and this equation decomposes into three independent Helmholtz equations
\begin{align}
  (\Delta + \varepsilon) E_i &= 0\,. 
\end{align}
From the Eikonal approximation $E_i = A_i \exp(i \phi(\mathbf{x}))$ it follows that
\begin{align}
  (\Delta + \varepsilon) \exp(i \phi) &= 0\,,
\end{align}
which means
\begin{align}
 i \Delta \phi - \partial_i \phi \partial_i \phi + \varepsilon &= 0\,.
\end{align}
Rewriting the equation into a more convenient form leads to (with $\phi = \sqrt{\varepsilon} \phi'$)
\begin{align}
 \partial_i \phi' \partial_i \phi' &= 1 + \frac{i}{\sqrt{\varepsilon}} \Delta \phi'\,.
\end{align}
For a further examination let $\phi' = \phi^\prime_0 + \frac{1}{\sqrt{\varepsilon}} \phi^\prime_1$ and neglecting higher order terms in $1/\sqrt{\varepsilon}$:
\begin{align}
 (\partial_i \phi^\prime_0)^2 + \frac{1}{\sqrt{\varepsilon}} 2 \partial_i \phi^\prime_0 \partial_i \phi^\prime_1 &= 1 + \frac{i}{\sqrt{\varepsilon}} \Delta \phi^\prime_0\,,
\end{align}
leads to
\begin{align}
 (\partial_i \phi^\prime_0)^2 &= 1\,,\\
 \partial_i \phi^\prime_0 \partial_i \phi^\prime_1 &= \frac{i}{2} \Delta \phi^\prime_0\,.
\end{align}
Now by using this approximative scheme we can extract further information for propagation of a raybundle through an isotropic medium.
Inserting the phase approximation into the divergence equation leads to:
\begin{align}
 \partial_i E_i &= A_i \partial_i \exp(i \phi) = i A_i (\partial_i \phi) \exp(i \phi) = i E_i (\partial_i \phi) = 0\,. 
\end{align}
This could also be used to neglect the divergence term after insertion of the phase approximation into the wave equation.


\section{Anisotropic evaluation}


For anisotropic media it follows from
\begin{align}
 \partial_i D_i &= 0\,,
\end{align}
with $D_i = \varepsilon_{ij} E_j$ that
\begin{align}
 \varepsilon_{ij} \partial_i E_j &= 0\,.
\end{align}
and therefore $i \varepsilon_{ij} (\partial_i \phi) E_j = 0$. Further from the wave equation
\begin{align}
 (\delta_{ij} \Delta - \partial_i \partial_j + \varepsilon_{ij}) E_j &= 0
\end{align}
and the phase approximation $E_i = A_i \exp(i \phi)$ it is seen that
\begin{align}
 (\delta_{ij} (i \Delta \phi - (\partial_k \phi)^2) - (i \partial_i \partial_j \phi + \partial_i \phi \partial_j \phi) + \varepsilon_{ij}) E_j &= 0
\end{align}
Now this equation only has non trivial solutions if the determinant of the matrix vanishes
\begin{align}
 \det(\delta_{ij} (i \Delta \phi - (\partial_k \phi)^2) 
 - (i \partial_i \partial_j \phi + \partial_i \phi \partial_j \phi)
 + \varepsilon_{ij}) &\stackrel{!}{=}0\,.
\end{align}



\end{document}
